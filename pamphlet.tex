\documentclass[oneside]{article}
% \documentclass[twoside]{article} %%

\usepackage{simplemath/core}
\usepackage{simplemath/pamphlet}

%opening
\title{Test}
\author{A. B.}

\begin{document}
\maketitle
\tableofcontents 
\newpage

Text Text Text

\Section[sec:einleitung]{Einleitung}
Mehr Text Text Text

\Subsection[subsec:motivation]{Motivation usw}
\begin{proof}
	$\{ x \in [a,b]: f(x) \neq g(x) \} = \emptyset $ \\
	\Implies $f = g$, also auch b.
	
	$\{ x \in [a,b]: f(x) \neq g(x) \} \neq \emptyset $ \\
	da $\UR$ laut der Erzählung von Grantelbär die Menge 
	$\big\{ x \in [a,b]: f(x) = g\big(f(x)\big) \big\}$ groß ist :D.\\
	
	Da diese keine Gerade \quote{bilden}.\\
	
	Außerdem $I^*(g) = I^*(f)$ und $I_*(g) = I_*(f)$ \footnote{Def. Riemann-integrierbar}.
	Und so ist auch $\int_{a}^{b} g(x) dx = \int_{a}^{b} f(x) dx$
\end{proof}

\Paragraph[par:bla]{Bla Bla Bla Paragraph}
Vielleicht
\begin{Eq*}
	a \limplies b
\end{Eq*}

\begin{ignore}
	Das hier wird ignoriert.
\end{ignore}

\Subsection{Subsektion}
\Subsubsection{Subsubsektion}
\Paragraph{Paragraph 3}
Text

\Image{0.35}{examplepicture}{Ein Bild}

Mehr Text

\Image[image:2]{0.35}{examplepicture}{Ein zweites Bild}

Mehr Text zu Abbildung \ref{image:2}.

Oder wie im Abschnitt \ref{subsec:motivation} erwähnt.

\end{document}
