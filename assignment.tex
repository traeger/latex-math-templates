\documentclass[]{article}

\usepackage{simplemath/core}
\usepackage{simplemath/assignment}

%opening
\title{Test}
\author{A. B.}

\begin{document}

\maketitle

\begin{task}{4}
	\begin{task-todo}
		\item Riemann integrierbar
		\item $\int_{a}^{b} f(x) dx$
	\end{task-todo}
	
	\begin{subtask}{a}
		\begin{proof}
			\case{1}
			$\{ x \in [a,b]: f(x) \neq g(x) \} = \emptyset $ \\
			\Implies $f = g$, also auch \reftask{b}.
			
			\case{2}
			$\{ x \in [a,b]: f(x) \neq g(x) \} \neq \emptyset $ \\
			da $\UR$ laut der Erzählung von Grantelbär die Menge 
			$\big\{ x \in [a,b]: f(x) = g\big(f(x)\big) \big\}$ groß ist :D.\\
			
			Da diese keine Gerade \quote{bilden}.\\
			
			Außerdem $I^*(g) = I^*(f)$ und $I_*(g) = I_*(f)$ \footnote{Def. Riemann-integrierbar}.
			Und so ist auch $\int_{a}^{b} g(x) dx = \int_{a}^{b} f(x) dx$
		\end{proof}
	\end{subtask}
\end{task}

\begin{task}{3}
	\begin{remark}
		Aus den Hauptsätzen $\dots$
	\end{remark}

	\begin{task-todo}
		\item Haben die Funktionen $f_i$ eine Stammfunktion?
	\end{task-todo}

	Vielleicht
	\begin{Eq*}
		a \limplies b
	\end{Eq*}

	\begin{ignore}
		Das hier wird ignoriert.
	\end{ignore}

	Text
	
	\Image{0.35}{examplepicture}{Ein Bild}
	
	Mehr Text
	
	\Image[image:2]{0.35}{examplepicture}{Ein zweites Bild}
	
	Mehr Text zu Abbildung \ref{image:2}.
\end{task}

\end{document}
