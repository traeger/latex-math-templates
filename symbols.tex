\documentclass[]{article}


\usepackage{environ}
\usepackage[utf8]{inputenc}
\usepackage[fleqn]{amsmath}
\usepackage{amsthm}
\usepackage{amssymb}
\usepackage{nameref}
\usepackage{enumitem}
\usepackage{babel}

% add a newline after each paragraph
\let\oldparagraph\paragraph
\renewcommand{\paragraph}[1]{\oldparagraph{#1}\mbox{}\\}

% no par intends
\setlength\parindent{0pt}

%ignore anything inside
\NewEnviron{ignore}[1]{
	
}

% write things aligned to the & symbol
\NewEnviron{Eq}{
	\begin{equation}\begin{aligned}
	\BODY
	\end{aligned}\end{equation}
}

\NewEnviron{Eq*}{
	\begin{equation*}\begin{aligned}
	\BODY
	\end{aligned}\end{equation*}
}

%basic definition and theorem environments
\newtheorem{definition}{Definition}[section]
\newtheorem{theorem}{Theorem}[section]
\newtheorem{corollary}{Korrolar}[theorem]
\newtheorem{lemma}[theorem]{Lemma}
\newtheorem*{remark}{Anmerkung}

\renewcommand*{\proofname}{Beweis}

% named definition and theorem environments
\NewEnviron{Definition}[1]{
	\begin{definition}[#1]
		\label{#1}
		\BODY
	\end{definition}
}

\NewEnviron{Theorem}[1]{
	\begin{theorem}[#1]
		\label{#1}
		\BODY
	\end{theorem}
}

\NewEnviron{Lemma}[1]{
	\begin{lemma}[#1]
		\label{#1}
		\BODY
	\end{lemma}
}

\NewEnviron{Korrolar}[1]{
	\begin{corollary}[#1]
		\label{#1}
		\BODY
	\end{corollary}
}

\NewEnviron{Remark}[1]{
	\begin{remark}[#1]
		\label{#1}
		\BODY
	\end{remark}
}

% remove the meaning of \proof and \endproof
\let\proof\relax
\let\endproof\relax
\NewEnviron{proof}{
	{\em Beweis}
	\BODY
	\hfill
	$\square$
}

% named reference
\def\Ref#1{\nameref{#1} (\ref{#1})}

% meta logic level
\def\Implies{\ensuremath{\mathbin{\Longrightarrow}}}
\def\RImplies{\ensuremath{\mathbin{\Longleftarrow}}}
\def\Iff{\ensuremath{\mathbin{\Longleftrightarrow}}}
\def\sep{\quad&}

% equation level
\renewcommand{\iff}{\leftrightarrow}

% sets
\def\setunion{\cup}
\def\Setunion{\bigcup\limits}
\def\setintersect{\cap}
\def\Setintersect{\bigcap\limits}
\def\setsize#1{\vert #1 \vert}

% set logic
\def\lxor{\mathbin{\dot{\lor}}}
\def\limplies{\mathop{\rightarrow}\limits}
\def\lrightimplies{\mathop{\leftarrow}\limits}
\def\liff{\mathop{\leftrightarrow}\limits}
\def\Land{\mathop{\bigwedge}\limits}
\def\Lor{\mathop{\bigvee}\limits}


% std set definitions
\def\UR{\ensuremath{\mathbin{\mathbb{R}}}}
\def\UN{\ensuremath{\mathbin{\mathbb{N}}}}
\def\UQ{\ensuremath{\mathbin{\mathbb{Q}}}}
\def\UZ{\ensuremath{\mathbin{\mathbb{Z}}}}
\def\UC{\ensuremath{\mathbin{\mathbb{C}}}}
\def\UB{\ensuremath{\mathbin{\mathbb{B}}}}
\def\neutral{\mathop{\mathbb{E}}}

% vectors and matices
\def\Vector#1{\begin{pmatrix}#1\end{pmatrix}}
\newenvironment{Matrix}{
	\begin{pmatrix}
}{
	\end{pmatrix}
}

% quoting
\renewcommand{\quote}[1]{\glqq #1\grqq{}}

% multirow
\NewEnviron{row}{
	\BODY
	\\[\baselineskip]
}
\NewEnviron{col}[2][t]{
	\begin{minipage}[#1]{#2\textwidth}
		\vspace{0pt}
		\BODY
	\end{minipage}
}

% special functions and symbols
\def\dd{\mathop{\text{d}\!}}

\def\newfunc#1{\mathop{\text{#1}}}
\def\newbinaryop#1{\mathbin{\text{#1}}}

\def\mod{\newbinaryop{mod}}
\def\ggT{\newfunc{ggT}}
\def\kgV{\newfunc{kgV}}
\def\Det{\newfunc{Det}}
\def\dim{\newfunc{dim}}
\def\min{\newfunc{min}}
\def\max{\newfunc{max}}

\usepackage{geometry}
\geometry{
	a4paper,
	left=20mm,
	top=20mm,
}

\NewEnviron{task}{
	\section{Aufgabe \thesection}
	\BODY
	\hfill
}

\NewEnviron{subtask}[1]{
	\subsection*{Aufgabe \thesection #1)}
	\BODY
}

\NewEnviron{task-todo}{
	Zu zeigen ist
	\begin{enumerate}[label=(\alph*)]
		\BODY
	\end{enumerate}
}

\newcommand{\case}[1]{\subparagraph{Fall #1}}
\newcommand{\reftask}[1]{(#1)}

\renewcommand{\thefootnote}{[\roman{footnote}]}




% for math-inline and tab support in verbatim mode
% Verbatim, PVerb
\usepackage{examplep}
\usepackage{fancyvrb}
\newcommand{\explain}{\qquad &}
\def\explained#1{#1 \qquad & \PVerb{#1}}

\begin{document}

\section{Mathe-Umgebung}

\paragraph{Inline}
Hallo $f = a$.
\begin{Verbatim}
	Hallo $f = a$.
\end{Verbatim}

\paragraph{Mehrzeilig}
\begin{Eq*}
	\sep f = a \\
	\Implies \sep a = b \\
\end{Eq*}
\begin{Verbatim}
	\begin{Eq*}
		\sep f = a \\
		\Implies \sep a = b \\
	\end{Eq*}
\end{Verbatim}

\paragraph{Mehrzeilig benannt}
\begin{Eq}
	\sep f = a \\
	\Implies \sep a = b \\
\end{Eq}
\begin{Verbatim}
	\begin{Eq}
		\sep f = a \\
		\Implies \sep a = b \\
	\end{Eq}
\end{Verbatim}

\section{Symbole}
\begin{row}[t]{0.5}
	\paragraph{Meta-Logik}
	\begin{Eq*}
		\explained{\Implies} \\
		\explained{\RImplies} \\
		\explained{\Iff} \\
	\end{Eq*}
\end{row}
\begin{row}[t]{0.5}
	\paragraph{Universen}
	\begin{Eq*}
		\explained{\UR} \\
		\explained{\UN} \\
		\explained{\UZ} \\
		\explained{\UQ} \\
		\explained{\UC} \\
		\explained{\UB}
	\end{Eq*}
\end{row}
\\

\begin{row}[t]{0.5}
	\paragraph{Logic}
	\begin{Eq*}
		\explained{\Land_x x \land y} \\
		\explained{\Lor_x x \lor y} \\
		\explained{\lnot x} \\
		\explained{x \limplies y} \\
		\explained{x \lrightimplies y} \\
		\explained{x \liff y} \\
		\explained{x \lxor y} \\
		\explained{\forall g: g} \\
		\explained{\exists g: g} \\
	\end{Eq*}
\end{row}
\begin{row}[t]{0.5}
	\paragraph{Mengen}
	\begin{Eq*}
		\explained{\emptyset} \\
		\explained{x \in A} \\
		\explained{x \notin A} \\
		\explained{\Setunion_x x \setunion y}\\
		\explained{\Setintersect_x x \setintersect y} \\
		\explained{a \subset b} \\
		\explained{a \subseteq b} \\
		\explained{a \subsetneq b} \\
		\explained{\setsize{A}} \\
		\explained{C = \{ a \in A \mid a \notin B \}} \\
		\explained{\partial A} \\
		\explained{\bar A} \\
		\explained{A_n = \{ 1 \dots n \}} \\
	\end{Eq*}
\end{row}
\\

\begin{row}[t]{0.5}
	\paragraph{Functions}
	\begin{Eq*}
		\explained{x \to y} \\
		\explained{x \mapsto y}\\
		\explained{f \circ g} \\
		\explained{f \ast g} \\
		\explained{\hat{f}} \\
	\end{Eq*}
\end{row}
\begin{row}[t]{0.5}
	\paragraph{Vergleiche}
	\begin{Eq*}
		\explained{a = b} \\
		\explained{a < b} \\
		\explained{a > b} \\
		\explained{a \leq b} \\
		\explained{a \geq b} \\
		\explained{a \neq b} \\
		\explained{a \equiv b} \\
		\explained{a \approx b} \\
		\explained{a \sim b} \\
	\end{Eq*}
\end{row}
\begin{row}[t]{0.5}
	\paragraph{Arithmetik}
	\begin{Eq*}
		\explained{\pm a} \\
		\explained{\lfloor a \rfloor} \\
		\explained{\lceil a \rceil} \\
		\explained{\sqrt{a + b}} \\
		\explained{\sqrt[3]{a + b}} \\
		\explained{x \cdot y} \\
		\explained{\sum_{x \in X} a + x} \\
		\explained{\sum_{i = x}^y a + i} \\
		\explained{\prod_{x \ in X} a + i} \\
		\explained{\min(a, b)} \\
		\explained{\max(a, b)} \\
	\end{Eq*}
\end{row}
\begin{row}[t]{0.5}
	\paragraph{Vectorräume}
	\begin{Eq*}
		\explained{x \times y} \\
		\explained{\Vector{1 \\ 2 \\ 3}}\\
		\explained{\begin{Matrix} 1 & 2 \\ 3 & 4 \end{Matrix}} \\
		\explained{\begin{Matrix} 1 & \dots \\ \vdots & b \end{Matrix}} \\
		\explained{\Det(x)} \\
		\explained{A + B} \\
		\explained{A * B} \\
		\explained{A \oplus B} \\
		\explained{A \otimes B} \\
		\explained{A / B} \\
		\explained{A^\perp} \\
		\explained{\langle A \rangle} \\
		\explained{\dim(A)}
	\end{Eq*}
\end{row}
\\

\begin{row}[t]{0.5}
	\paragraph{Lina \& AZ}
	\begin{Eq*}
		\explained{a \mod b} \\
		\explained{a \mid b} \\
		\explained{a \nmid b} \\
		\explained{a \parallel b} \\
		\explained{a \perp b} \\
		\explained{\ggT(x, y)} \\
		\explained{\kgV(x, y)} \\
		\explained{\big[ x \big] } \\
		\explained{\neutral} \\
	\end{Eq*}
\end{row}
\begin{row}[t]{0.5}
	\paragraph{Ana}
	\begin{Eq*}
		\explained{\dd x} \\
		\explained{\frac{\dd f}{\dd x}} \\
		\explained{\frac{\partial f}{\partial x}} \\
		\explained{\int x \dd x} \\		
		\explained{\int_0^\infty x \dd x} \\
		\explained{\big[ x \big]_0^y} \\
		\explained{\lim_{x \nearrow a} f(x)} \\
		\explained{\lim_{x \searrow a} f(x)} \\
		\explained{\lim_{x \to a} f(x)} \\
		\explained{f^\prime} \\
		\explained{f^{\prime\prime}} \\
		\explained{\dot f} \\
		\explained{\ddot f} \\
		\explained{\nabla f}
	\end{Eq*}
\end{row}
\\

\section{Layout}
\begin{Eq*}
	\explained{f(x) = 
		\begin{cases}
			1 & x = 0 \\
			0 & \text{sonst}
		\end{cases}
	} \\
\end{Eq*}

\section{Weiteres}
\def\bin{\newbinaryop{bin}}
\def\fn{\newfunc{fn}}

\paragraph{Einen eigenen Binär-Operator definieren}
Am Anfang des Dokuments definieren: 
\begin{Eq*}
	\PVerb{\def\bin{\newbinaryop{bin}}}
\end{Eq*}
Dann kann dieser wie folgt genutzt werden:

\begin{Eq*}
	\explained{a \bin b} \\
\end{Eq*}

\paragraph{Einen eigenen Funktion definieren}
Am Anfang des Dokuments definieren: 
\begin{Eq*}
	\PVerb{\def\fn{\newfunc{fn}}}
\end{Eq*}

Dann kann dieser wie folgt genutzt werden:
\begin{Eq*}
	\explained{\fn(a, b)} \\
\end{Eq*}

\end{document}
