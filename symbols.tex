\documentclass[]{article}

\usepackage{simplemath/core}
\usepackage{simplemath/assignment}

% for math-inline and tab support in verbatim mode
% Verbatim, PVerb
\usepackage{examplep}
\usepackage{fancyvrb}
\newcommand{\explain}{\qquad &}
\def\explained#1{#1 \qquad & \PVerb{#1}}

\begin{document}

\section{Mathe-Umgebung}

\paragraph{Inline}
Hallo $f = a$.
\begin{Verbatim}
	Hallo $f = a$.
\end{Verbatim}

\paragraph{Mehrzeilig}
\begin{Eq*}
	\sep f = a \\
	\Implies \sep a = b \\
\end{Eq*}
\begin{Verbatim}
	\begin{Eq*}
		\sep f = a \\
		\Implies \sep a = b \\
	\end{Eq*}
\end{Verbatim}

\paragraph{Mehrzeilig benannt}
\begin{Eq}
	\sep f = a \\
	\Implies \sep a = b \\
\end{Eq}
\begin{Verbatim}
	\begin{Eq}
		\sep f = a \\
		\Implies \sep a = b \\
	\end{Eq}
\end{Verbatim}

\section{Symbole}
\begin{row}
	\begin{col}{0.2}
		\begin{Eq*}
			\explained{\alpha} \\ 
			\explained{\beta} \\ 
			\explained{\chi} \\ 
			\explained{\delta} \\
			\explained{\epsilon} \\ 
			\explained{\eta} \\ 
			\explained{\gamma} \\ 
			\explained{\iota} \\ 
			\explained{\kappa} \\ 
		\end{Eq*}
	\end{col}
	\begin{col}{0.2}
		\begin{Eq*}
			\explained{\lambda} \\ 
			\explained{\mu} \\ 
			\explained{\nu} \\ 
			\explained{o} \\
			\explained{\omega} \\
			\explained{\phi} \\ 
			\explained{\pi} \\ 
			\explained{\psi} \\ 
			\explained{\rho} \\ 
		\end{Eq*}
	\end{col}
	\begin{col}{0.2}
		\begin{Eq*}
			\explained{\sigma} \\ 
			\explained{\tau} \\ 
			\explained{\theta} \\ 
			\explained{\upsilon} \\ 
			\explained{\xi} \\ 
			\explained{\zeta} \\ 
			\explained{\digamma} \\ 
			\explained{\varepsilon} \\ 
			\explained{\varkappa} \\ 
		\end{Eq*}
	\end{col}
	\begin{col}{0.2}
		\begin{Eq*}
			\explained{\varphi} \\ 
			\explained{\varpi} \\ 
			\explained{\varrho} \\ 
			\explained{\varsigma} \\ 
			\explained{\vartheta} \\ 
			\explained{\Delta} \\ 
			\explained{\Gamma} \\ 
			\explained{\Lambda} \\ 
			\explained{\Omega} \\
		\end{Eq*}
	\end{col}
	\begin{col}{0.2}
		\begin{Eq*}
			\explained{\Phi} \\ 
			\explained{\Pi} \\ 
			\explained{\Psi} \\ 
			\explained{\Sigma} \\ 
			\explained{\Theta} \\
			\explained{\Upsilon} \\
			\explained{\Xi} \\
		\end{Eq*}
	\end{col}
\end{row}

\begin{row}
	\begin{col}{0.5}
		\paragraph{Meta-Logik}
		\begin{Eq*}
			\explained{\Implies} \\
			\explained{\RImplies} \\
			\explained{\Iff} \\
		\end{Eq*}
	\end{col}
	\begin{col}{0.5}
		\paragraph{Universen}
		\begin{Eq*}
			\explained{\UR} \\
			\explained{\UN} \\
			\explained{\UZ} \\
			\explained{\UQ} \\
			\explained{\UC} \\
			\explained{\UB}
		\end{Eq*}
	\end{col}
\end{row}

\begin{row}
	\begin{col}{0.5}
		\paragraph{Logic}
		\begin{Eq*}
			\explained{\Land_x x \land y} \\
			\explained{\Lor_x x \lor y} \\
			\explained{\lnot x} \\
			\explained{x \limplies y} \\
			\explained{x \lrightimplies y} \\
			\explained{x \liff y} \\
			\explained{x \lxor y} \\
			\explained{\forall g: g} \\
			\explained{\exists g: g} \\
		\end{Eq*}
	\end{col}
	\begin{col}{0.5}
		\paragraph{Mengen}
		\begin{Eq*}
			\explained{\emptyset} \\
			\explained{x \in A} \\
			\explained{x \notin A} \\
			\explained{\Setunion_x x \setunion y}\\
			\explained{\Setintersect_x x \setintersect y} \\
			\explained{a \subset b} \\
			\explained{a \subseteq b} \\
			\explained{a \subsetneq b} \\
			\explained{\setsize{A}} \\
			\explained{C = \{ a \in A \mid a \notin B \}} \\
			\explained{\partial A} \\
			\explained{\bar A} \\
			\explained{A_n = \{ 1 \dots n \}} \\
		\end{Eq*}
	\end{col}
\end{row}

\begin{row}
	\begin{col}{0.5}
		\paragraph{Functions}
		\begin{Eq*}
			\explained{x \to y} \\
			\explained{x \mapsto y}\\
			\explained{f \circ g} \\
			\explained{f \ast g} \\
			\explained{\hat{f}} \\
		\end{Eq*}
	\end{col}
	\begin{col}{0.5}
		\paragraph{Vergleiche}
		\begin{Eq*}
			\explained{a = b} \\
			\explained{a < b} \\
			\explained{a > b} \\
			\explained{a \leq b} \\
			\explained{a \geq b} \\
			\explained{a \neq b} \\
			\explained{a \equiv b} \\
			\explained{a \approx b} \\
			\explained{a \sim b} \\
		\end{Eq*}
	\end{col}
\end{row}

\begin{row}
	\begin{col}{0.5}
		\paragraph{Arithmetik}
		\begin{Eq*}
			\explained{\pm a} \\
			\explained{\lfloor a \rfloor} \\
			\explained{\lceil a \rceil} \\
			\explained{\sqrt{a + b}} \\
			\explained{\sqrt[3]{a + b}} \\
			\explained{x \cdot y} \\
			\explained{\sum_{x \in X} a + x} \\
			\explained{\sum_{i = x}^y a + i} \\
			\explained{\prod_{x \ in X} a + i} \\
			\explained{\min(a, b)} \\
			\explained{\max(a, b)} \\
		\end{Eq*}
	\end{col}
	\begin{col}{0.5}
		\paragraph{Vectorräume}
		\begin{Eq*}
			\explained{x \times y} \\
			\explained{\Vector{1 \\ 2 \\ 3}}\\
			\explained{\begin{Matrix} 1 & 2 \\ 3 & 4 \end{Matrix}} \\
			\explained{\begin{Matrix} 1 & \dots \\ \vdots & b \end{Matrix}} \\
			\explained{\Det(x)} \\
			\explained{A + B} \\
			\explained{A * B} \\
			\explained{A \oplus B} \\
			\explained{A \otimes B} \\
			\explained{A / B} \\
			\explained{A^\perp} \\
			\explained{\langle A \rangle} \\
			\explained{\dim(A)}
		\end{Eq*}
	\end{col}
\end{row}

\begin{row}
	\begin{col}{0.5}
		\paragraph{Lina \& AZ}
		\begin{Eq*}
			\explained{a \mod b} \\
			\explained{a \mid b} \\
			\explained{a \nmid b} \\
			\explained{a \parallel b} \\
			\explained{a \perp b} \\
			\explained{\ggT(x, y)} \\
			\explained{\kgV(x, y)} \\
			\explained{\big[ x \big] } \\
			\explained{\neutral} \\
		\end{Eq*}
	\end{col}
	\begin{col}{0.5}
		\paragraph{Ana}
		\begin{Eq*}
			\explained{\dd x} \\
			\explained{\frac{\dd f}{\dd x}} \\
			\explained{\frac{\partial f}{\partial x}} \\
			\explained{\int x \dd x} \\		
			\explained{\int_0^\infty x \dd x} \\
			\explained{\big[ x \big]_0^y} \\
			\explained{\lim_{x \nearrow a} f(x)} \\
			\explained{\lim_{x \searrow a} f(x)} \\
			\explained{\lim_{x \to a} f(x)} \\
			\explained{f^\prime} \\
			\explained{f^{\prime\prime}} \\
			\explained{\dot f} \\
			\explained{\ddot f} \\
			\explained{\nabla f}
		\end{Eq*}
	\end{col}
\end{row}

\section{Layout}
\begin{Eq*}
	\explained{f(x) = 
		\begin{cases}
			1 & x = 0 \\
			0 & \text{sonst}
		\end{cases}
	} \\
\end{Eq*}

\section{Weiteres}
\def\bin{\newbinaryop{bin}}
\def\fn{\newfunc{fn}}

\paragraph{Einen eigenen Binär-Operator definieren}
Am Anfang des Dokuments definieren: 
\begin{Eq*}
	\PVerb{\def\bin{\newbinaryop{bin}}}
\end{Eq*}
Dann kann dieser wie folgt genutzt werden:

\begin{Eq*}
	\explained{a \bin b} \\
\end{Eq*}

\paragraph{Einen eigenen Funktion definieren}
Am Anfang des Dokuments definieren: 
\begin{Eq*}
	\PVerb{\def\fn{\newfunc{fn}}}
\end{Eq*}

Dann kann dieser wie folgt genutzt werden:
\begin{Eq*}
	\explained{\fn(a, b)} \\
\end{Eq*}

\end{document}
